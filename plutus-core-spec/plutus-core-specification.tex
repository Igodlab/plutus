% Plutus Core Specification
\title{Formal Specification of\\the Plutus Core Language (version 2.0) }
\date{24th April 2019}

\documentclass[a4paper]{article}

\usepackage{blindtext, graphicx}
\usepackage{url}

% *** MATH PACKAGES ***
%
\usepackage[cmex10]{amsmath}
\usepackage{stmaryrd}

% *** ALIGNMENT PACKAGES ***
%
\usepackage{array}
\usepackage{float}  %% Try to improve placement of figures.  Doesn't work well with subcaption package.
\usepackage{subcaption}
\usepackage{caption}

% Stuff for splitting figures over page breaks
\DeclareCaptionLabelFormat{continued}{#1~#2 (Continued)}
\captionsetup[ContinuedFloat]{labelformat=continued}


% correct bad hyphenation here
\hyphenation{}

\usepackage{subfiles}
\usepackage{geometry}
\usepackage{pdflscape}
\usepackage[title]{appendix}


% *** IMPORTS FOR PLUTUS LANGUAGE ***

\usepackage[T1]{fontenc}
\usepackage{bussproofs,amsmath,amssymb}
\usepackage{xcolor}
\usepackage{alltt}

\usepackage{listings}
\lstset{
  basicstyle=\ttfamily,
  columns=fullflexible,
  mathescape=true,
  escapeinside={|}{|}   %% Inside listings you can say things like |\textit{blah blah}|
}




% *** DEFINITIONS FOR PLUTUS LANGUAGE ***

%%% General Misc. Definitions

\newcommand{\red}[1]{\textcolor{red}{#1}}
\newcommand{\redfootnote}[1]{\red{\footnote{\red{#1}}}}
\newcommand{\blue}[1]{\textcolor{blue}{#1}}
\newcommand{\bluefootnote}[1]{\blue{\footnote{\blue{#1}}}}


\newcommand{\diffbox}[1]{\text{\colorbox{lightgray}{\(#1\)}}}
\newcommand{\judgmentdef}[2]{\fbox{#1}

\vspace{0.5em}

#2}
\newcommand{\hyphen}{\operatorname{-}}
\newcommand{\repetition}[1]{\overline{#1}}
\newcommand{\Fomega}{F$^{\omega}$}
\newcommand{\keyword}[1]{\texttt{#1}}
\newcommand{\construct}[1]{\texttt{(} #1 \texttt{)}}



%%% Term Grammar

\newcommand{\sig}[3]{[#1](#2)#3}
\newcommand{\constsig}[1]{#1}
\newcommand{\con}[1]{\construct{\keyword{con} ~ #1}}
\newcommand{\abs}[3]{\construct{\keyword{abs} ~ #1 ~ #2 ~ #3}}
\newcommand{\inst}[2]{\texttt{\{}#1 ~ #2\texttt{\}}}
\newcommand{\lam}[3]{\construct{\keyword{lam} ~ #1 ~ #2 ~ #3}}
\newcommand{\app}[2]{\texttt{[} #1 ~ #2 \texttt{]}}
\newcommand{\iwrap}[3]{\construct{\keyword{iwrap} ~ #1 ~ #2 ~ #3}}
\newcommand{\wrap}{\iwrap}
%% ^ Temporary fix to avoid substituting all occurences of new keyword
\newcommand{\unwrap}[1]{\construct{\keyword{unwrap} ~ #1}}
\newcommand{\builtin}[3]{\construct{\keyword{builtin} ~ #1 ~ #2 ~ #3}}
\newcommand{\error}[1]{\construct{\keyword{error} ~ #1}}



%%%  Type Grammar

\newcommand{\funT}[2]{\construct{\keyword{fun} ~ #1 ~ #2}}
\newcommand{\ifixT}[2]{\construct{\keyword{ifix} ~ #1 ~ #2}}
\newcommand{\fixT}{\ifixT}
%% ^ Temporary fix to avoid substituting all occurences of new keyword
\newcommand{\allT}[3]{\construct{\keyword{all} ~ #1 ~ #2 ~ #3}}
\newcommand{\conIntegerType}[1]{\keyword{integer}}
\newcommand{\conBytestringType}[1]{\keyword{bytestring}}
\newcommand{\builtinT}[2]{\construct{\keyword{builtin} ~ #1 ~ #2}}
\newcommand{\conT}[1]{\construct{\keyword{con} ~ #1}}
\newcommand{\lamT}[3]{\construct{\keyword{lam} ~ #1 ~ #2 ~ #3}}
\newcommand{\appT}[2]{\texttt{[} #1 ~ #2 \texttt{]}}

\newcommand{\typeK}{\construct{\keyword{type}}}
\newcommand{\funK}[2]{\construct{\keyword{fun} ~ #1 ~ #2}}



%%% Program Grammar

\newcommand{\version}[2]{\construct{\keyword{program} ~ #1 ~ #2}}



%%% Judgments

\newcommand{\hypJ}[2]{#1 \vdash #2}
\newcommand{\ctxni}[2]{#1 \ni #2}
\newcommand{\validJ}[1]{#1 \ \operatorname{valid}}
\newcommand{\termJ}[2]{#1 : #2}
\newcommand{\typeJ}[2]{#1 :: #2}
\newcommand{\istermJ}[2]{#1 : #2}
\newcommand{\istypeJ}[2]{#1 :: #2}



%%% Contextual Normalization

\newcommand{\ctxsubst}[2]{#1\{#2\}}
\newcommand{\typeStep}[2]{#1 ~ \rightarrow_{ty} ~ #2}
\newcommand{\typeMultistep}[2]{#1 ~ \rightarrow_{ty}^{*} ~ #2}
\newcommand{\typeBoundedMultistep}[3]{#2 ~ \rightarrow_{ty}^{#1} ~ #3}
\newcommand{\step}[2]{#1 ~ \rightarrow ~ #2}
\newcommand{\multistepIndexed}[3]{#1 ~ \rightarrow^{#2} ~ #3}
\newcommand{\normalform}[1]{\lfloor #1 \rfloor}
\newcommand{\subst}[3]{[#1/#2]#3}
\newcommand{\kindEqual}[2]{#1 =_{\mathit{k}} #2}
\newcommand{\typeEqual}[2]{#1 =_{\mathit{ty}} #2}
\newcommand{\typeEquiv}[2]{#1 \equiv_{\mathit{ty}} #2}


\newcommand{\inConTFrame}[1]{\conT{#1}}
\newcommand{\inAppTLeftFrame}[1]{\appT{\_}{#1}}
\newcommand{\inAppTRightFrame}[1]{\appT{#1}{\_}}
\newcommand{\inFunTLeftFrame}[1]{\funT{\_}{#1}}
\newcommand{\inFunTRightFrame}[1]{\funT{#1}{\_}}
\newcommand{\inAllTFrame}[2]{\allT{#1}{#2}{\_}}
\newcommand{\inFixTLeftFrame}[1]{\fixT{\_}{#1}}
\newcommand{\inFixTRightFrame}[1]{\fixT{#1}{\_}}
\newcommand{\inLamTFrame}[2]{\lamT{#1}{#2}{\_}}

\newcommand{\inBuiltin}[5]{\builtin{#1}{#2}{#3 #4 #5}}



%%% CK Machine Normalization

\newcommand{\cksteps}[2]{#1 ~\mapsto~ #2}
\newcommand{\ckforward}[2]{#1 \triangleright #2}
\newcommand{\ckbackward}[2]{#1 \triangleleft #2}
\newcommand{\ckerror}{\blacklozenge}

\newcommand{\inInstLeftFrame}[1]{\inst{\_}{#1}}
\newcommand{\inWrapRightFrame}[2]{\iwrap{#1}{#2}{\_}}
\newcommand{\inUnwrapFrame}{\unwrap{\_}}
\newcommand{\inAppLeftFrame}[1]{\app{\_}{#1}}
\newcommand{\inAppRightFrame}[1]{\app{#1}{\_}}

% These are for use inside listings and $...$.  If you just use
% \textit in listings it uses the italic tt font and the spacing
% inside the words is a bit strange.  Spacing is also bad if you
% just put something like "integer" in math text.
\newcommand\unit{\ensuremath{\mathit{unit}}}
\newcommand\one{\ensuremath{\mathit{one}}}
\newcommand\boolean{\ensuremath{\mathit{boolean}}}
\newcommand\integer{\ensuremath{\mathit{integer}}}
\newcommand\bytestring{\ensuremath{\mathit{bytestring}}}
\newcommand\str{\ensuremath{\mathit{str}}}
\newcommand\true{\ensuremath{\mathit{true}}}
\newcommand\false{\ensuremath{\mathit{false}}}
\newcommand\case{\ensuremath{\mathit{case}}}
\newcommand\signed{\ensuremath{\mathit{signed}}}
\newcommand\txhash{\ensuremath{\mathit{txhash}}}
\newcommand\pubkey{\ensuremath{\mathit{pubkey}}}
\newcommand\blocknum{\ensuremath{\mathit{blocknum}}}
%% \newcommand\uniqmem[1]{#1^{\blacktriangledown}}  %% Unique member of size type



\begin{document}
%
% paper title
% can use linebreaks \\ within to get better formatting as desired

%% 6.0: "User-friendly" version with lots of explication for the reader.
%% 16th April 2019

\maketitle

\thispagestyle{plain}
\pagestyle{plain}


%\begin{abstract}
%\boldmath
%The Plutus Language is outlined, together with the major
%design decisions for implementations. A formal specification of the
%language is given, including an elaborator and bidirectional type
%system.

\section{Plutus Core}

Plutus Core is a typed, eagerly evaluated $\lambda$-calculus designed
for use as a transaction validation language in blockchain
systems. More precisely, Plutus Core is the system $F_\omega$ of
Girard and Reynolds (see \cite{Girard-thesis},
\cite{Reynolds-type-structure}, \cite[\S30]{Pierce:TAPL}) with a
number of extensions (isorecursive types, higher kinds, and a library
of basic types and functions).  Plutus Core is meant to be a
compilation target, and this is reflected in the design of the
language: while writing large Plutus Core programs by hand is
difficult, the language is relatively straightforward to formalise
with a proof assistant.

We have tried to keep the Plutus Core language small. There is no
explicit support for algebraic datatypes, but they are representable
using, for example, the Scott encoding (\cite{Scott-encoding}: see
also \cite{Koopman:2014} and ~\cite{Geuvers-2014} for example; note that
the Scott encoding coincides with the Church encoding for
non-recursive types). Similarly, there are no explicit constructs for
recursion or branching, but the type system is sufficiently expressive
to allow us to use standard combinators (for example the $Z$
combinator and Church/Scott encoded booleans) for such purposes.


\subsection{Blockchain issues}
Plutus Core code (and code intended for execution on a blockchain in
general) can be executed in two different environments:
\textit{off-chain} and \textit{on-chain}.  As the name suggests,
off-chain execution doesn't happen on the blockchain itself, but in
some other environment, such as in an electronic wallet on a
smartphone or PC.  In contrast, on-chain execution takes place on
\textit{core nodes}, machines which are actually maintaining the
blockchain.

It is important that core nodes process transactions quickly and
efficiently; moreover, core nodes benefit from executing transactions
and compete with each other to do so.  Denial of service thus becomes
an issue for on-chain code, and it is important that a malicious user
cannot submit code which consumes excessive amounts of processor time
or memory. To deal with this problem, a charge is levied when a core
node runs on-chain code.  Units of so-called \textit{gas} are consumed
as a program runs, and users must pay a fee in advance to cover gas
costs.  If at any point during execution the amount of gas consumed
exceeds the amount than has been paid for then the program is
immediately terminated and the fee is forfeited.

Resource limitations preclude complicated on-chain code analysis
and verification.  We perform type checking prior to execution of
Plutus Core Code, but no other validation.  Type checking itself takes
time and requires memory, so there is also a gas charge for on-chain
type checking.

%% Because of these issues, it is important that the cost of on-chain
%% code execution is easy to calculate, certainly during execution (so
%% that gas consumption can be accounted for), but ideally also in
%% advance so that a user has some idea of how much a contract is likely
%% to cost and can make sure that they buy enough gas to allow their
%% contract to run to completion.  The design of Plutus Core incorporates
%% a number of features to facilitate cost analysis for this purpose.


\section{Syntax}

The grammar of Plutus Core is given in
Figures~\ref{fig:lexical_grammar} and
\ref{fig:grammar}. This grammar describes the abstract
syntax trees of Plutus Core in a convenient notation which can also be
used as concrete syntax for purposes such as experimentation and
debugging.

Lexemes are described in standard regular expression notation.  The
following lexemes are reserved as keywords for use in concrete syntax:
\texttt{abs}, \texttt{all}, \texttt{builtin}, \texttt{bytestring},
\texttt{con}, \texttt{error}, \texttt{fun}, \texttt{ifix},
\texttt{integer}, \texttt{iwrap}, \texttt{lam}, \texttt{program},
\texttt{type}, \texttt{unwrap}.  The only other lexemes
are round brackets $\texttt{(}\ldots\texttt{)}$, square brackets
$\texttt{[}\ldots\texttt{]}$, and braces
$\texttt{\{}\ldots\texttt{\}}$.  Spaces and tabs are allowed anywhere,
and have no effect save to separate lexemes.

Application in both terms and types is indicated by square
brackets, and instantiation in terms is indicated by braces. We
permit the use of multi-argument application and instantiation as
syntactic sugar for iterated application.
For instance,
\[
  [M_0 ~ M_1 ~ M_2 ~ M_3]
  \]
\noindent is short for
\[
  [[[M_0 ~ M_1] ~ M_2] ~ M_3]
\]
All subsequent definitions assume iterated application and instantiation
has been expanded out, and use only the binary form. Implementations
may use the multi-argument forms.


\subfile{figures/LexicalGrammar}

\subfile{figures/Grammar}

\newcommand\fixtype[1]{\mu\,\alpha.#1}  % Just for talking about the fix operator in the notes.

\subsection{Notes on the grammar}
\paragraph{Terms.}
\begin{itemize}
\item $\con{cn}$ represents a constant belonging to some built-in
  type: for example \texttt{(con 4)} is an integer with the value 127.
  See Section~\ref{sec:builtins} for more on this.
\item $\abs{\alpha}{K}{V}$ represents a polymorphic value abstracted
  over a type; this would often be donated by $\Lambda\alpha{::}K.V$.
  Note that we require the body of the abstraction to be a
  \textit{value}, not a term: see the note on the value restriction below.
\item $\inst{M}{A}$ represents a polymorphic term instantiated at a particular type.
\item $\iwrap{A}{B}{M}$ and $\unwrap{M}$: see the note on recursive types below.
\item $\lam{x}{A}{M}$ is standard lambda-abstraction, $\lambda{}x{:}{A}.{M}$.
\item $\app{M}{N}$ is standard function application.
\item $\builtin{bn}{A^*}{M^*}$ is application of a built-in function: see Section~\ref{sec:builtins} for more information.
\item $\error{A}$ causes an error, terminating computation immediately.
\end{itemize}

%% Re the value restriction:  Phil says
% Value restriction: This guarantees that one can erase a type
% abstraction to its body (otherwise, one needs to erase a type
% abstraction to a value abstraction). It also corresponds to the value
% restriction in Standard ML, which ensures polymorphism works with
% effects. For instance, if we wanted to translate Plutus into a monad
% to make non-termination and reading the environment explicit, that is
% easy with the value restriction; otherwise, it is (so far as I know)
% not known how to do it. The restriction appears in my paper Theorems
% for Free for Free with Ahmed, Jamner, and Siek in ICFP 2017.



\paragraph{Values.} Values are terms which cannot undergo any further reduction.

\paragraph{Types.} Plutus Core contains a copy of the simply typed lambda calculus
  at the type level, together with a few extensions.
\begin{itemize}
\item $\funT{A}{B}$ is the type of functions from $A$ to $B$, $A \rightarrow B$.
\item $\allT{\alpha}{K}{A}$ represents the type of a polymorphic term (eg a type abstraction), $\forall \alpha{::}K.A$.
\item $\fixT{A}{B}$ represents a recursive type: see the note in Section~\ref{sec:ifix-note} below for more information.
\item $\lamT{\alpha}{K}{A}$ is abstraction of types over types, $\lambda \alpha{::}K.A$.
\item $\appT{A}{B}$ is function application at the type level.
\item $\conT{tcn}$ represents a built-in type: for example, $\conT{\textrm{integer}}$
is the type of integers.  See Section~\ref{sec:builtins} for more information.
\end{itemize}

\paragraph{Type Values.} All types can be reduced to a type value (or
  \textit{normalised type}) which cannot undergo any further
  reduction: see section~\ref{sec:reduction}.  The type normalisation
  process always terminates: the proof is similar to the proof of the
  fact that reduction of terms in the simply typed lambda calculus
  always terminates (\cite[\S12]{Pierce:TAPL}).  The type-level
  structure of Plutus Core is slightly more complicated than the
  simply typed lambda calculus, but the extra forms (\texttt{ifix},
  \texttt{all}, built-in base types) don't provide any new
  opportunities for type reduction and thus have little effect on the
  proof.

\paragraph{Kinds.} In Plutus Core we have a copy of the 
simply typed lambda calculus at the type level. The types of the
simply typed lambda calculus are lifted to the level of
\textit{kinds} in Plutus Core, allowing the type system to talk about
operations which themselves occur at the level of types.
The basic kind (usually written as $\star$) is denoted by \texttt{(type)}.

\paragraph{Signatures.}  A signature
$$[\alpha_1::K_1, \ldots, \alpha_m::K_m](t_1, \ldots, t_n)u$$ represents
the type
$$\forall \alpha_1::K_1\,.\, \ldots \,.\, \forall \alpha_m::K_m\,.\, t_1
\rightarrow \ldots \rightarrow t_n \rightarrow u.$$
Both $m$ and $n$
may be zero, in which case the corresponding list is empty.
Signatures are not used in the concrete syntax of Plutus Core, but we
do use them later in this document, in typing rules and in the
description of built-in values and functions.

\paragraph{Programs and Versions.} A complete Plutus Core program 
consists of a standard version number (1.0.2, for example) indicating
the Plutus Core version and a \textit{closed} term (i.e., a term with no
free variables) forming the body of the program.  The version number
is used by the Plutus Core evaluator and other tools to check that
they are dealing with code conforming to the correct version of the
Plutus Core language.

\subsection{More detailed remarks}
\subsubsection{Type Normalisation}
 Type checking requires normalisation to be performed, for example to
 check that two types are equal.  This process can be expensive, so we
 require that code supplied for on-chain execution comes with all type
 annotations pre-normalised.  This can be done during off-chain type
 checking.  Another issue here is that normalisation can cause the
 size of a type to increase (in the worst case, exponentially), and
 requiring normalisation to be performed before a program is submitted
 to the chain reduces on-chain gas costs.

%% \subsubsection{Sizes and built-in types and functions}
%% \label{sec:size-note}
%% Details of built-in types and functions are given
%% in Section~\ref{sec:builtins}, but a remark on sizes may be helpful at this
%% point.  Plutus Core currently provides two types of built-in data with
%% variable sizes: integers and bytestrings (blocks of binary data used
%% for cryptographic purposes).  There is no fixed upper bound on the
%% size of values of these types, but -- as mentioned in the introduction -- it
%% could be problematic if users were allowed to create values of arbitrary size 
%% at runtime. To avoid this, sizes are tracked in the type system so that 
%% appropriate charges can be made. 

%% For integers, this works as follows (the system for 
%% bytestrings is similar).  Integers are signed values:
%% there is no fixed upper bound, but values are represented as sequences
%% of a specified number of 8-bit bytes.  For example, \texttt{1~!~100}
%% is the integer 100 stored as a single byte.  Literal constants must
%% fit into the specified size: \texttt{1~!~200} gives a compile-time
%% error. If a value overflows at runtime then an error occurs and
%% execution is terminated.  Integers are manipulated using built-in
%% functions (see Figure~\ref{fig:builtins}) whose types
%% include the size. This prevents users from writing programs which
%% generate arbitrarily large values during execution; it also allows us
%% to charge according to the actual amount of computation performed,
%% which will increase as the size of the integers involved increases.


\subsubsection{Recursive types}
\label{sec:ifix-note}
\noindent
The operator \texttt{ifix} allows one to define recursive types in
Plutus Core: \texttt{ifix A B} is a type such that \texttt{ifix A B
  $\cong$ A (ifix A) B} where \texttt{A} is a \textit{pattern
  functor}~\cite[2.4]{backhouseetal98} and \texttt{B} is an
index. Pattern functors that \texttt{ifix} receives bind two variables:
one for building recursive occurrences and the other to act as an
index. Indices can be used in order to get parameterised data types,
but also in order to control the shape of the data type: depending on
an index, recursive occurrences can be instantiated differently.  This
allows us to encode a wide variety of data types, including non-regular and
mutually recursive data types\blue{\footnote{\blue{We could do with more detail (or perhaps an example) here, since
  it'll be much more difficult for the reader to work out what's going
  on than with standard \texttt{fix}.  With luck we'll soon have a publication
  to refer to.}}}.

Note that we have an \textit{isomporhism} of types here rather than
equality: fixpoint types are
\textit{isorecursive}~\cite[20.2]{Pierce:TAPL} with explicit maps

$$
\texttt{iwrap} : \texttt{A (fix A) B} \rightarrow \texttt{ifix A B}
$$

\noindent and

$$
\texttt{unwrap} : \texttt{ifix A B} \rightarrow  \texttt{A (ifix A) B}
$$

\noindent such that

$$
\texttt{iwrap} \circ \texttt{unwrap} = id_{ \texttt{ifix A B}}
\mbox{\qquad and \qquad}
\texttt{unwrap} \circ \texttt{iwrap} = id_{ \texttt{A (ifix A) B}}
$$
%% I really hate not having a full stop after the second equation, but it looks weird.

\noindent The use of isorecursive types makes it somewhat more difficult to
write \textit{terms}, but makes it much easier to reason about
\textit{types}, and in particular simplifies the typechecking process
considerably.


\subsubsection{The value restriction}
In $\abs{\alpha}{K}{V}$ we require the body $V$ to be a value. We will
refer to this restriction as the \textit{value restriction} because it
can be regarded as a generalisation of the value restriction in
Standard ML (\cite[22.7]{Pierce:TAPL}). Experience has shown that
allowing general terms as bodies of type abstractions can cause a
number of difficulties.  For example, Standard ML's value restriction
is required to avoid a number of problems when (implicit) type
abstractions include non-values; other issues are explored
in~\cite[2.4]{Ahmed:2017}. Plutus Core's restriction to values avoids
these problems and is not very onerous in practice.
  
\section{Type Correctness}

We define for Plutus Core a number of typing judgments which explain
ways that a program can be well-formed. First, in Figure
\ref{fig:contexts}, we define the grammar of contexts as
contexts appear here for the first time. Contexts are sequences of
variables. In System F there are two different sorts of variables:
term variables and type variables. For succinctness of presentation we
keep information about both sorts of variables in the same
context. Type variables carry their name and a kind and term variables
carry their name and a type. The rules for context validity explain
how valid contexts can be constructed, the empty context is valid,
given valid context $\Gamma$, it can be extended with a fresh variable
and kind or a fresh variable and a valid type in $\Gamma$. Note we do
not need a separate judgement for valid kinds. They are so simple that
any kind that can be constructed according to the grammar is a valid
kind.

%We also consider variable judgements in Figure
%\ref{fig:contexts}. Technically this judgement is defined
%mutually with the typing and kinding judgement defined in Figures
%\ref{fig:type_synthesis} and
%\ref{fig:kind_synthesis}. We give axiomatic rules for term
%and type variables at the right-hand end of the context. For term and
%type variables defined further to the left in the context the
%judgement is dependent on a judgement about a shorter context. These
%two styles of rules for all combinations of term and type variables
%allow us to give judgements for variables in all positions in the
%context. This style of presentation is analogous to how one might
%define instrinsically typed/kinded de Bruijn indices.

%% ---------------- Contexts ---------------- %%

\begin{figure}[H]
{
    \[\begin{array}{lrclr}
        \textrm{Ctx} & \Gamma  & ::= & \epsilon                    & \textrm{empty context} \\
                     &         &     & \Gamma, \typeJ{\alpha}{K}   & \textrm{type variable} \\
                     &         &     & \Gamma, \termJ{x}{A}        & \textrm{term variable} \\
    \end{array}\]

    \judgmentdef{\(\validJ{\Gamma}\)}{Context $\Gamma$ is valid}

    \begin{prooftree}
        \AxiomC{}
        \UnaryInfC{\(\validJ{\epsilon}\)}
    \end{prooftree}

    \begin{prooftree}
        \AxiomC{\(\validJ{\Gamma}\)}
        \AxiomC{$\alpha$ is free in $\Gamma$}
        \BinaryInfC{\(\validJ{\Gamma, \typeJ{\alpha}{K}}\)}
    \end{prooftree}

    \begin{prooftree}
        \AxiomC{\(\validJ{\Gamma}\)}
        \AxiomC{$x$ is free in $\Gamma$}
        \AxiomC{\(\hypJ{\Gamma}{\istypeJ{A}{\typeK{}}}\)}
        \TrinaryInfC{\(\validJ{\Gamma, \termJ{x}{A}}\)}
    \end{prooftree}

%    \judgmentdef{\(\ctxni{\Gamma}{J}\)}{In valid context $\Gamma$ we can make judgements about variables where $J$ is either a typing or a kinding judgement.}
%
%    \begin{prooftree}
%        \AxiomC{}
%        \UnaryInfC{\(\ctxni{\Gamma, \typeJ{\alpha}{K}}{\typeJ{\alpha}{K}}\)}
%    \end{prooftree}
%
%    \begin{prooftree}
%        \AxiomC{}
%        \UnaryInfC{\(\ctxni{\Gamma, \termJ{x}{A}}{\termJ{x}{A}}\)}
%    \end{prooftree}
%
%    \begin{prooftree}
%        \AxiomC{\(\ctxni{\Gamma}{\typeJ{\alpha}{K}}\)}
%        \AxiomC{\(\alpha \not= \beta\)}
%        \BinaryInfC{\(\ctxni{\Gamma, \typeJ{\beta}{J}}{\typeJ{\alpha}{K}}\)}
%    \end{prooftree}
%
%     \begin{prooftree}
%        \AxiomC{\(\ctxni{\Gamma}{\typeJ{\alpha}{K}}\)}
%        \UnaryInfC{\(\ctxni{\Gamma, \termJ{y}{T}}{\typeJ{\alpha}{K}}\)}
%    \end{prooftree}
%
%    \begin{prooftree}
%        \AxiomC{\(\ctxni{\Gamma}{\termJ{x}{A}}\)}
%        \UnaryInfC{\(\ctxni{\Gamma, \typeJ{\beta}{J}}{\termJ{x}{A}}\)}
%    \end{prooftree}
%
%     \begin{prooftree}
%        \AxiomC{\(\ctxni{\Gamma}{\termJ{x}{A}}\)}
%        \AxiomC{\(x \not= y\)}        
%        \BinaryInfC{\(\ctxni{\Gamma, \termJ{y}{B}}{\termJ{x}{A}}\)}
%    \end{prooftree}


}
    \captionof{figure}{Contexts}
    \label{fig:contexts}
\end{figure}

\newpage 
\noindent Figure \ref{fig:kind_synthesis} defines what
it means for a type to synthesise a kind. Plutus Core is a
higher-kinded version of System F, so we have a number of standard
System F rules (tyvar,tyall,tyfun) together with some extensions with
extensions to higher kinds (tylam,tyapp) and to indexed
recursive types (tyfix). We also introduce builtin types (tycon) to
support integers and bytestrings.

%% ---------------- Kind synthesis ---------------- %%

\begin{figure}[H]
{
    \judgmentdef{\(\hypJ{\Gamma}{\istypeJ{A}{K}}\)}{In valid context $\Gamma$, type $A$ has kind $K$}

    \begin{prooftree}
        \AxiomC{\(({\typeJ{\alpha}{K}}) \in \Gamma\)}
        \RightLabel{tyvar}
        \UnaryInfC{\(\hypJ{\Gamma}{\istypeJ{\alpha}{K}}\)}
    \end{prooftree}

    \begin{prooftree}
        \AxiomC{\(\hypJ{\Gamma, \typeJ{\alpha}{K}}{\istypeJ{A}{\typeK{}}}\)}
        \RightLabel{tyall}
        \UnaryInfC{\(\hypJ{\Gamma}{\istypeJ{\allT{\alpha}{K}{A}}{\typeK{}}}\)}
    \end{prooftree}

    \begin{prooftree}
		\AxiomC{\(\hypJ{\Gamma}{\istypeJ{B}{K}}\)}
		\AxiomC{\(\hypJ{\Gamma}{\istypeJ{A}{\funK{\funK{K}{\typeK{}}}{\funK{K}{\typeK{}}}}}\)}
        \RightLabel{tyfix}
        \BinaryInfC{\(\hypJ{\Gamma}{\istypeJ{\fixT{A}{B}}{\typeK{}}}\)}
    \end{prooftree}

    \begin{prooftree}
        \AxiomC{\(\hypJ{\Gamma}{\istypeJ{A}{\typeK{}}}\)}
        \AxiomC{\(\hypJ{\Gamma}{\istypeJ{B}{\typeK{}}}\)}
        \RightLabel{tyfun}
        \BinaryInfC{\(\hypJ{\Gamma}{\istypeJ{\funT{A}{B }}{\typeK{}}}\)}
    \end{prooftree}

    \begin{prooftree}
        \AxiomC{\(\hypJ{\Gamma, \typeJ{\alpha}{J}}{\istypeJ{A}{K}}\)}
        \RightLabel{tylam}
        \UnaryInfC{\(\hypJ{\Gamma}{\istypeJ{\lamT{\alpha}{J}{A}}{\funK{J}{K}}}\)}
    \end{prooftree}

    \begin{prooftree}
        \AxiomC{\(\hypJ{\Gamma}{\istypeJ{A}{\funK{J}{K}}}\)}
        \AxiomC{\(\hypJ{\Gamma}{\istypeJ{B}{J}}\)}
        \RightLabel{tyapp}
        \BinaryInfC{\(\hypJ{\Gamma}{\istypeJ{\appT{A}{B}}{K}}\)}
    \end{prooftree}

    \begin{prooftree}
        \AxiomC{$tcn$ is a type constant in in Figure \ref{fig:type_constants}}
        \RightLabel{tycon}
        \UnaryInfC{\(\hypJ{\Gamma}{\istypeJ{\conT{tcn}}{\typeK{}}}\)}
    \end{prooftree}

    \captionof{figure}{Kind Synthesis}
    \label{fig:kind_synthesis}
}
\end{figure}


\noindent In Figure \ref{fig:type_synthesis}, we define the type
synthesis judgment, which explains how a term synthesises a type. We
have rules analogous to STLC (var,lam,app), extensions to System F
(abs,inst). Higher-kinding introduces computation in types so we need
the rule conv. Iso-recursive types introduce terms for wrapping and
unwrapping recursive values (wrap,unwrap), and to support constants
and builtins we have con, builtin and error.

%% ---------------- Type synthesis ---------------- %%

\begin{figure}[H]
    \judgmentdef{\(\hypJ{\Gamma}{\istermJ{M}{A}}\)}{In valid context $\Gamma$, term $M$ has type $A$}

    \begin{prooftree}
        \AxiomC{\(({\termJ{x}{A}}) \in \Gamma\)}
        \RightLabel{var}
        \UnaryInfC{\(\hypJ{\Gamma}{\istermJ{x}{A}}\)}
    \end{prooftree}

    \begin{prooftree}
        \AxiomC{$cn$ has constant signature $\constsig{tcn}$ in Figure \ref{fig:constants}}
        \RightLabel{con}
        \UnaryInfC{\(\hypJ{\Gamma}{\istermJ{cn}{\conT{tcn}}}\)}
    \end{prooftree}

    \begin{prooftree}
        \AxiomC{\(\hypJ{\Gamma, \typeJ{\alpha}{K}}{\istermJ{M}{B}}\)}
        \RightLabel{abs}
        \UnaryInfC{\(\hypJ{\Gamma}{\istermJ{\abs{\alpha}{K}{M}}{\allT{\alpha}{K}{B}}}\)}
    \end{prooftree}

    \begin{prooftree}
        \AxiomC{\(\hypJ{\Gamma}{\istermJ{L}{C}}\)}
        \AxiomC{\(\typeEquiv{C}{\allT{\alpha}{K}{B}}\)}
        \AxiomC{\(\hypJ{\Gamma}{\istypeJ{A}{K}}\)}
        \RightLabel{inst}
        \TrinaryInfC{\(\hypJ{\Gamma}{\istermJ{\inst{L}{A}}{\subst{A}{\alpha}{B}}}\)}
    \end{prooftree}

    \begin{prooftree}
    	\AxiomC{\(\hypJ{\Gamma}{\istypeJ{B}{K}}\)}
		\alwaysNoLine
		\UnaryInfC{\(\hypJ{\Gamma}{\istypeJ{A}{\funK{\funK{K}{\typeK{}}}{\funK{K}{\typeK{}}}}}\)}
		\UnaryInfC{\(\hypJ{\Gamma}{\istermJ{M}{C}}\)}
		\UnaryInfC{\(\typeEquiv{C}{\appT{\appT{A}{\lamT{\beta}{K}{\fixT{A}{\beta}}}}{B}}\)}
		\alwaysSingleLine
    	\RightLabel{wrap}
        \UnaryInfC{\(\hypJ{\Gamma}{\istermJ{\wrap{A}{B}{M}}{\fixT{A}{B}}}\)}
    \end{prooftree}

    \begin{prooftree}
    	\AxiomC{\(\hypJ{\Gamma}{\istermJ{M}{C}}\)}
		\AxiomC{\(\typeEquiv{C}{\fixT{A}{B}}\)}
		\AxiomC{\(\hypJ{\Gamma}{\istypeJ{B}{K}}\)}
		\RightLabel{unwrap}
        \TrinaryInfC{\(\hypJ{\Gamma}{\istermJ{\unwrap{M}}{\appT{\appT{A}{\lamT{\beta}{K}{\fixT{A}{\beta}}}}{B}}}\)}
    \end{prooftree}

    \begin{prooftree}
        \AxiomC{\(\hypJ{\Gamma}{\istypeJ{A}{\typeK{}}}\)}
        \AxiomC{\(\hypJ{\Gamma, \termJ{y}{A}}{\istermJ{M}{B}}\)}
        \RightLabel{lam}
        \BinaryInfC{\(\hypJ{\Gamma}{\istermJ{\lam{y}{A}{M}}{\funT{A}{B}}}\)}
    \end{prooftree}

    \begin{prooftree}
        \AxiomC{\(\hypJ{\Gamma}{\istermJ{L}{C}}\)}
        \AxiomC{\(\typeEquiv{C}{\funT{A}{B}}\)}
        \AxiomC{\(\hypJ{\Gamma}{\istermJ{M}{A'}}\)}
        \AxiomC{\(\typeEquiv{A}{A'}\)}
        \RightLabel{app}
        \QuaternaryInfC{\(\hypJ{\Gamma}{\istermJ{\app{L}{M}}{B}}\)}
    \end{prooftree}

    \begin{prooftree}
        \alwaysNoLine
        \AxiomC{$bn$ has signature $\sig{\alpha_0 :: K_0, ..., \alpha_m :: K_m}{B_0, ..., B_n}{C}$ in Figure \ref{fig:builtins}}
        \UnaryInfC{\(\hypJ{\Gamma}{\istermJ{M_i}{D_i}}\)}
        \UnaryInfC{\(\typeEquiv{D_i}{\subst{A_0, ..., A_m}{\alpha_0, ..., \alpha_m}{B_i}}\)}
        \alwaysSingleLine
        \RightLabel{builtin}
        \UnaryInfC{\(\hypJ{\Gamma}{\istermJ{\builtin{bn}{A_0 ... A_m}{M_0 ... M_n}}{\subst{A_0, ..., A_m}{\alpha_0, ..., \alpha_m}{C}}}\)}
    \end{prooftree}

    \begin{prooftree}
        \AxiomC{\(\hypJ{\Gamma}{\istypeJ{A}{\typeK{}}}\)}
        \RightLabel{error}
        \UnaryInfC{\(\hypJ{\Gamma}{\istermJ{\error{A}}{A}}\)}
    \end{prooftree}
    
    \begin{prooftree}
    	\AxiomC{\(\hypJ{\Gamma}{\istermJ{M}{A}}\)}
		\AxiomC{\(\typeEquiv{A}{A'}\)}
		\RightLabel{conv}
		\BinaryInfC{\(\hypJ{\Gamma}{\istermJ{M}{A'}}\)}
    \end{prooftree}

    \captionof{figure}{Type Synthesis}
    \label{fig:type_synthesis}

\end{figure}

\noindent Finally, type synthesis for builtins are elaborated in tabular form
rather than in inference rule form, in Figure
\ref{fig:builtins}, which also gives the reduction
semantics. This table also specifies what conditions trigger an error.


\section{Reduction and Execution}
\label{sec:reduction}

In this section we define a standard eager,
small-step contextual semantics~\cite[5.3]{Harper:PFPL} (or
\textit{reduction semantics}~\cite[\S2]{Felleisen-Hieb}) for Plutus
Core in terms of the reduction relation for types
(\(\typeStep{A}{A'}\)) (Figure~\ref{fig:type-reduction}) 
and terms (\(\step{M}{M'}\)) (Figure~\ref{fig:term-reduction}), which
incorporates both $\beta$ reduction and contextual congruence. We make
use of the transitive closure of these stepping relations via the
usual Kleene star notation.
% It seems that Harper talks about "contextual semactics" in the 
% first edition of his book, but changed it to "contextual dymanics"
% in the second.  It doesn't seem to be a standard term elsewhere though.

In the context of a blockchain system, it can be useful to also have a
step indexed version of stepping, indicated by a superscript count of
steps (\(\multistepIndexed{M}{n}{M'}\)). In order to prevent
transaction validation from looping indefinitely, or from simply
taking an inordinate amount of time, which would be a serious
flaw in the blockchain system, we can use step indexing to put an
upper bound on the number of computational steps that a program can
have. In this setting, we would pick some upper bound $\mathit{max}$
and then perform steps of terms $M$ by computing which $M'$ is such
that \(\multistepIndexed{M}{\mathit{max}}{M'}\).



\subsection{Type reduction}
Because the Plutus Core type system contains a copy of the simply
typed lambda calculus, complex computations can take place at the
level of types.  Reductions in the type system always transform a
given type into an equivalent type, and so have no effect on the term
level.  In conjunction with the fact that type reduction always
terminates, this allows us to perform normalisation (i.e., reduction to
a form which cannot undergo any further reduction) statically, before
execution begins.  Figure~\ref{fig:type-reduction} contains the rules
for type reduction, and Figure~\ref{fig:type-equivalence} contains
rules for type equivalence.

\subfile{figures/TypeReduction}


\newpage
\subsection{Term reduction}
Execution of Plutus Core programs is performed by (possibly
non-terminating) reduction of well-typed terms.  The reduction rules
are contained in Figure~\ref{fig:term-reduction}, and give us a fairly
standard operational semantics.  

\subfile{figures/TermReduction}

\subsection{An abstract machine for evaluating Plutus Core programs}
This section contains a description of an abstract machine for
executing Plutus Core.  This is based on the CK machine of Felleisen
and Friedman~\cite{Felleisen-CK-CEK}. 

This machine is intended as a reference implementation which is
amenable to formalisation, ideally so that it can be proved to
implement the operational semantics described above.  The CK machine
is inefficient because it implements application as $\beta$-reduction,
so that when evaluating $(\lambda x.M)V$, $V$ must be substituted
bodily for $x$ wherever it occurs in $M$.  This process can require
considerable amounts of time and space.  More efficient machines are
available (for example the CEK machine from~\cite{Felleisen-CK-CEK},
which uses environments binding actual arguments to variables), and in
practice we would use something based on such a machine; however, the
CK machine will still provide a useful bridge for formalisation
purposes.

\subfile{figures/CkMachine}

\noindent The machine alternates between two main phases: the
\textit{compute} phase ($\triangleright$), where it recurses down
the AST looking for values, saving surrounding contexts as frames (or
\textit{reduction contexts}) on a stack as it goes; and the 
\textit{return} phase ($\triangleleft$), where it has obtained a value and
pops a frame off the stack to tell it how to proceed next.  In
addition there is an error state $\blacklozenge$ which halts execution
with an error, and a stop state $\square$ which halts execution and
returns a value to the outside world.

To evaluate a program $\texttt{(program } v\ M \texttt{)}$, we first
check that the version number $v$ is valid, then start the machine in
the state $\cdot \triangleright M$.  It can be proved that the
transitions in Figure~\ref{fig:ck_machine} always preserve
validity of states, so that the machine can never enter a state such as
  $\cdot \triangleleft M$
or
$s, \texttt{(unwrap \_)} \triangleleft \texttt{(lam }x\ A \ M\texttt{)}$
which isn't covered by the rules.  If such a
situation were to occur in an implementation then it would indicate
that the machine was incorrectly implemented or that it was attempting
to evaluate an invalid program which should have been detected during
parsing or typechecking.


\section{Built in types, functions, and values}
\label{sec:builtins}
Plutus Core comes with a pre-defined set of built-in types and
functions which will be useful for blockchain applications.  The set
of built-ins is fixed, although in future we may provide a framework
to allow customisation for specialised blockchains.

As mentioned earlier, there are two basic types: \texttt{integer} and
\texttt{bytestring}.  These types are given in
Figure~\ref{fig:type_constants}: for example \texttt{(con
  integer)} is the type of signed integers.
We provide standard arithmetic and comparison operations for integers
and a number of list-like functions for bytestrings. The details are
given in Figure~\ref{fig:builtins}, using a number of
abbreviations given in Figure~\ref{fig:type_abbreviations}.


Note the following:
\begin{itemize}
\item The built-in functions in Figure~\ref{fig:builtins}
are all monomorphic, and hence the type arguments are all empty: []. 
It is possible that we may add some polymorphic built-in functions
at some time in the future.
\item Some of the entries in Figure~\ref{fig:builtins}
  contain \textit{success conditions}.  If a success condition is
  violated then the program terminates immediately and returns
  the value \texttt{(error)}.
\item We provide two versions of the division and remainder operations
  for integers.  These differ in their treatment of negative
  arguments\footnote{The standard mathematical definition (the
  so-called \textit{division algorithm}, or \textit{Euclidean
  division}) is as follows: if $n,d \in \mathbb{Z}$ and $d \ne 0$ then
  there exist unique integers $q, r \in \mathbb{Z}$ such that $n=qb+r$
  and $0 \le r < \lvert d \rvert$; $q$ is the \textit{quotient} and
  $r$ is the \textit{remainder}.  Neither of our sets of operations
  implements this definition! This could conceivably be problematic if we
  ever need to do number-theoretic calculations.}:

  \begin{itemize}
  
  \item \texttt{divideInteger} and \texttt{modInteger} implement the
    standard mathematical integer division
    operations: \texttt{divideInteger} rounds downwards and the sign
    of \texttt{modInteger} $n$ $d$ is the same as the sign of $d$.
    These correspond to Haskell's \texttt{div} and \texttt{mod}
    operators.

  \item \texttt{quotientInteger} and \texttt{remainderInteger}
    represent the operations found in many computer languages and
    CPUs: \texttt{quotientInteger} rounds towards zero and the sign
    of \texttt{remainderInteger $n$ $d$} is the same as the sign of
    $n$.  These correspond to Haskell's \texttt{rem} and \texttt{quot}
    operators.  \end{itemize}

  
\item We use fixed Scott encodings for certain types and values, specifically for
  the \textit{unit} and \textit{boolean} types.  Compilers targeting
  Plutus Core must be aware of these encodings and use them
  appropriately.
\end{itemize}
\subfile{figures/Builtins}

\section{Transaction Validation}
The primary use for Plutus Core is for executing code on blockchain
nodes; specifically, it is intended for scripts used to validate
transactions taking place on the chain.  This section will give a
brief description of the background which will be useful in a
subsequent example.

The Cardano blockchain uses a variant of Bitcoin's \textit{Unspent
  Transaction Output} (UTXO) model.  Transactions on the chain consume
inputs and produce outputs consisting of cryptocurrency, and outputs
from a transaction can be spent as inputs to a later transaction.
Users do not have an account containing currency, but instead have the
right to spend outputs from previous transactions which have not yet
been spent (UTXOs).

To spend an output, a user has to prove that they have the right to do
so.  In a simple case, an output might have been locked with a public
key and a user would have to use the corresponding private key to
unlock it.  However, more complicated scenarios are possible: for
example, a specified set of users might have to act together to spend
a particular output, or an output might not be spendable until a
certain period of time has passed since its production.

The process of verifying that an output can be spent is called
\textit{validation}, and is performed by executing \textit{scripts},
which in our case will be Plutus Core programs.


Cardano uses a validation strategy which we call the \textit{Extended
  UTXO model}.  A full specification of the Extended UTXO model is
currently being prepared: in the meantime the fundamental ideas can be
found in~\cite{Zahnentferner18-Chimeric} and
\cite{Zahnentferner18-UTxO}.

The basic idea is that when a transaction wishes $T$ to consume an
unspent output $U$, it applies a script called the \textit{validator}
to another script called the \textit{redeemer}\footnote{The Extended
  UTXO model also involves another script called the \textit{data
    script}, but for simplicity we won't discuss this here}; the
validator also has access to certain parts of the blockchain state,
such as the current block number (\textit{blocknum}) and the hash
\textit{txhash} of the current transaction.  If the result of applying
the validator to the redeemer is the Plutus Core \texttt{error} term
then validation fails and the output cannot be spent; otherwise
validation succeeds and the output can be spent.

The job of the validator is to check whether the transaction $T$ is
allowed to consume the output $U$, and the redeemer provides evidence
that $U$ can in fact be spent.  For example, if a user $X$ wishes to
spend an output $U$ then $X$ might provide a redeemer consisting of a
signature obtained from $U$ with a private key belonging to $X$ and
the validator would use $X$'s public key to check that the signature
is valid.  We consider this example in more detail in the next
section.

\section{Example}

\subfile{figures/ValidatorExample}

%\section{Erasure}

%TO WRITE

%\subfile{figures/PlutusCoreErasureGrammar}

%\subfile{figures/PlutusCoreErasureReduction}

%\subfile{figures/PlutusCoreErasureTheorem}

%\section{Example}

%\subfile{figures/PlutusCoreExampleAgain}

\begin{appendices}
\section{Unrestricted Algorithmic Type System}

For implementation purposes it is useful to have a variant of the type
system that is more algorithmic in its presentation. We give this
here, showing the figures that have been changed (relative to the
declarative version above), and highlighting the specific parts of
each rule that is different, where possible (Figures~\ref{fig:contexts_algorithmic_unrestricted}
and~\ref{fig:type_synthesis_algorithmic_unrestricted}) .

\subfile{figures/TypeSynthesis-Algorithmic-Unrestricted.tex}

\newpage

\section{Restricted Algorithmic Type System}

For blockchain purposes, it is useful to not only have an algorithmic
system, but to require that no un-normalised types are present in
programs, so as to reduce the amount of computation necessary to
typecheck a program. It's also useful to have a way of bounding the
amount of computation that can be done in type checking, for security
purposes. To that end, we present a restricted grammar and type
reduction, and a type system that reflects these.
Figures~\ref{fig:grammar_algorithmic_restricted}--\ref{fig:type_synthesis_algorithmic_restricted}
are additions with respect to the unrestricted version, and show only
the new or changed figures and their respective highlighted
differences.

\subfile{figures/Grammar-Algorithmic-Restricted.tex}

\subfile{figures/Reduction-Algorithmic-Restricted.tex}

\subfile{figures/TypeSynthesis-Algorithmic-Restricted.tex}

\end{appendices}

\bibliographystyle{plain} %% ... or whatever
\bibliography{plutus-core-specification}

\end{document}
