\documentclass[../plutus-core-specification.tex]{subfiles}

\begin{document}

We illustrate the use of Plutus Core by constructing a simple
validator program for the situation described in the previous
section. We present components of this program in a high-level style:
for instance, we write
\begin{lstlisting}
   |\one| : |\unit|
   |\one| = (abs $a$ (type) (lam $x$ $a$ $x$))
\end{lstlisting}
for the element of the type $unit$ (defined Figure~\ref{fig:Plutus_core_type_abbreviations}).

We stress that declarations in this style are not part of the Plutus
Core language. We merely use the familiar syntax to present our
example. If the high-level definitions in our example were compiled to
a Plutus Core expression, it would result in something like Figure
\ref{fig:Continuized_Let_Example}.

We proceed by defining the booleans. Like \textit{unit}, the type \textit{boolean}
below is an abbreviation in the specification. Some built-in constants
return values of type $boolean$. When needed, user programs should
contain the declarations below. We have

% Note that |\boolean| below causes \boolean to be treated as normal latex.
% It works like this because of the \escapeinside setting in \lstset in the document header.

\begin{lstlisting}
  |\true| : |\boolean|
  |\true| = (abs $a$ (type)
          (lam $x$ $a$
          (lam $y$ $a$
                $x$ )))
\end{lstlisting}
and similarly
\begin{lstlisting}
  |\false| : |\boolean|
  |\false| = (abs $a$ (type)
           (lam $x$ $a$
           (lam $y$ $a$
                 $y$ )))
\end{lstlisting}

\noindent Next, we define the \case{} function for the type $boolean$ as follows:
\begin{lstlisting}
  |\case| : (all $a$ (type)
          (fun |\boolean|
          (fun (fun |\unit| $a$) (fun (fun |\unit| $a$) $a$))))
  |\case| = (abs $a$ (type)
          (lam $b$ |\boolean|
          (lam $t$ (fun |\unit| $a$)
          (lam $f$ (fun |\unit| $a$)
             [
               [ {$b$ (fun |\unit| $a$)} $t$ $f$ ]
               |\one|
             ]
          ))))
\end{lstlisting}
The reader is encouraged to verify that
\begin{lstlisting}
  [{|\case| a} |\true| (lam |\unit| u x) (lam |\unit| u y)] $\stackrel{*}{\rightarrow}$ x
\end{lstlisting}
and
\begin{lstlisting}
  [{|\case| a} |\false| (lam |\unit| u x) (lam |\unit| u y)] $\stackrel{*}{\rightarrow}$ y
\end{lstlisting}

\noindent We can use \case{} to define the following function:
\begin{lstlisting}
  verifyIdentity :
    (fun (con bytestring) (fun (con bytestring) unit))
  verifyIdentity =
    (lam |\pubkey| (con bytestring)
    (lam |\signed| (con bytestring)
  [ { |\case| |\unit| } [ (builtin verifySignature) |\signed| |\txhash| |\pubkey| ]
        (lam u |\unit| |\one|)
        (lam u |\unit| (error |\unit|))
      ]))
\end{lstlisting}
the idea being that the first argument is a public key, and the second
argument is the result of signing the hash of the current transaction
(contained in $\mathit{txhash} : \texttt{(con bytestring)}$) with
the corresponding private key\footnote{In practice $\mathit{txhash}$
  is contained in a data structure supplied as an extra parameter to
  the validator script at the start of the validation process.  The
  Plutus Core code required to access $\mathit{txhash}$ is a (rather
  complicated) implementation detail: we omit it here, regarding
  $\mathit{txhash}$ as a value contained in some enclosing
  environment.}.  The function terminates normally if and only if the
signature is valid, returning \texttt{error} otherwise. Now, given
Alice's public key we can apply our function to obtain one that
verifies whether or not its input is the result of Alice signing the
current block number. Again, we stress that the Plutus Core expression
corresponding to \texttt{verifyIdentity} is going to look something
like Figure \ref{fig:Continuized_Let_Example}.

% The next bit no longer makes a lot of sense because 

%% With minimal modification we might turn our function into one that
%% verifies a signature of the current block number; specifically, we
%% could replace \txhash{} with
%% \begin{lstlisting}
%%   [ {intToByteString (con 16) (con 32)}
%%     256
%%     [{|\blocknum| (con 16)} 16]
%%   ]
%% \end{lstlisting}

%% Notice that we must supply \blocknum{} with the size we wish to use to
%% store the result twice, once at the type level and again at the term
%% level. This is necessary because we want to have the size information
%% available both at the type level, to facilitate gas calculations, and
%% at the term level, so that once types are erased the runtime will know
%% how much memory to allocate. This quirk is present in a number of the
%% built in functions.

\begin{figure*}[h]  % Using H here causes undefined references to this figure
\begin{lstlisting}
(lam |\pubkey| (con bytestring)
(lam |\signed| (con bytestring)
    [ 
    {
        (abs $a$ (type)
        (lam $b$ (all $a$ (type) (fun $a$ (fun $a$ $a$)))
            (lam $t$ (fun (all $a$ (type) (fun $a$ $a$)) $a$)
            (lam $f$ (fun (all $a$ (type) (fun $a$ $a$)) $a$)
                [
                [ { $b$ (fun (all $a$ (type) (fun $a$ $a$)) $a$) } $t$ $f$ ]
                (abs $a$ (type) (lam $x$ $a$ $x$))
                ]
            )
            )
        )
        )
        (all $a$ (type) (fun $a$ $a$))
    }
    [ (builtin verifySignature) |\signed| |\txhash| |\pubkey| ]
    (lam $u$ (all $a$ (type) (fun $a$ $a$))
        (abs $a$ (type) (lam $x$ $a$ $x$))
    )
    (lam $u$ (all $a$ (type) (fun $a$ $a$))
        (error (all $a$ (type) (fun $a$ $a$)))
    )
    ]
)
)
\end{lstlisting}
\caption{Example of Section 5 written out in full}
\label{fig:Continuized_Let_Example}
\end{figure*}

\end{document}

