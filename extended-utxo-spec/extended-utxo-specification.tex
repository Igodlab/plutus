%% Extended UTXO Specification

\newcommand\version{-1}


\documentclass[a4paper]{article}

\usepackage{blindtext, graphicx}
\usepackage{url}

% *** MATHS PACKAGES ***
%
\usepackage[cmex10]{amsmath}
\usepackage{amssymb}
\usepackage{stmaryrd}
\usepackage{amsthm}

% *** ALIGNMENT PACKAGES ***
%
\usepackage{array}
\usepackage{float}  %% Try to improve placement of figures.  Doesn't work well with subcaption package.
\usepackage{subcaption}
\usepackage{caption}

% correct bad hyphenation here
\hyphenation{}

\usepackage{subfiles}
\usepackage{geometry}
\usepackage{listings}
\usepackage{xcolor}
\usepackage{verbatim}


%%% General Misc. Definitions

\newcommand{\red}[1]{\textcolor{red}{#1}}
\newcommand{\redfootnote}[1]{\red{\footnote{\red{#1}}}}
\newcommand{\blue}[1]{\textcolor{blue}{#1}}
\newcommand{\bluefootnote}[1]{\blue{\footnote{\blue{#1}}}}

\renewcommand{\i}{\textit}  % Just to speed up typing: replace these in the final version
\renewcommand{\t}{\texttt}  % Just to speed up typing: replace these in the final version

\theoremstyle{definition}  %% This applies to all following \newtheorem items
\newtheorem{definition}{Definition}
\newtheorem{example}{Example}

%% ------------- Start of document ------------- %%

\begin{document}
\title{Formal Specification of the Extended UTxO Model\\ Version \version}
\maketitle

\thispagestyle{plain}
\pagestyle{plain}

\section{Introduction: The Extended UTXO Model}
\label{sec:intro}
The Cardano blockchain uses a variant of the \textit{Unspent
  Transaction Output} (UTXO) model used by Bitcoin.  Transactions
consume \textit{unspent outputs} (UTXOs) from previous transactions and
produce new outputs which can be used as inputs to later transactions.

\subsection{Outputs and scripts}
\label{sec:scripts}
There is no well-defined notion of ownership for UTXOs.  In many
transactions an output will accrue to a single user who is then
entitled to spend it at a later date.  However, in general the notion
of ownership is more complex: an output of a contract might require
the cooperation of several users before it could be spent, or it might
not be spendable until some other condition has been met (for example
a certain period of time may have to pass).  At the extremes, an
output could be spendable by anyone, or by no-one.

In order to deal with this complexity, an output can be locked by a
\textit{script}, requiring another script to unlock it.  In the
Cardano setting, scripts are Plutus Core
programs~\cite{Plutus-Core-spec} (equivalently, expressions).  In the
basic model, each input to a transaction comes with a \i{validator}
script which checks that the transaction is allowd to spend the
output. To prove that it is allowed to spend an output, the
transaction supplies another script, called the \i{redeemer}\footnote{
  The validator plays a role similar to that of BitCoin's
  \texttt{scriptPubKey} and the redeemer to \texttt{scriptSig}.}.

In a process known as \i{validation}, the validator is run with the
redeemer as input, and if it terminates without an error arising then
the output can be spent.

A simple example of this is a \i{pay-to-pubkey} script, where the
redeemer consists of a signature for the current transaction produced
using a private key belonging to the owner of the output.  The
validator (provided by the owner of the output) would check the
signature using a known public key: if the public key corresponds to
the private key then validation succeeds, otherwise it fails.  Thus
the output can only be spent by the owner of the relevant public key.

\section{The Extended UTXO Model}

The papers~\cite{Zahnentferner18-UTxO}
and~\cite{Zahnentferner18-Chimeric} contain a mathematical description
of the UTXO model currently used in Cardano, which is basically that
described in Section~\ref{sec:intro}.  In this document we specify
some extensions which increase the expressivity of the existing UTXO
model.  We refer to the extended version as the \textit{Extended UTXO}
(XUTXO) model.  The XUTXO model introduces the following new features:

\begin{itemize}
\item Every transaction has a \textit{validity interval}, consisting
  of two slot numbers: a core node will only process the transaction
  if the current slot number lies within the transaction's validity
  interval.
\item We introduce a new type of script called a \i{data script},
  and each unspent output has a (possibly empty) data script
  associated with it.  The introduction of the data script
  increases the expresssivity of the model considerably.
\item During validation, validator scripts are provided with
  information about the current state of the blockchain.  This
  includes things such as the current slot number, the validity
  interval of the current transaction, and the hash of the
  transaction.  This enables us to ensure that the validation process
  is \i{deterministic}, meaning that the cost of validation can be
  calculated before a transaction is submitted to the blockchain.
  This makes it much simpler for users to calculate transaction fees.
\end{itemize}
  

\noindent\blue{We require more exposition here to motivate these extensions.}

\section{A Formal Description of the Extended UTXO Model}
\label{section:extended-utxo}

In this section we give a formal description of the XUTXO model.  The
description is given in a straightforward set-theoretic form, which
(a) admits an almost direct translation into Haskell, and (b) should
easily be amenable to mechanical formalisation.  This will potentially
allow us to argue formally about smart contracts and to develop tools
for automatic contract analysis.

A validator script is a function of three arguments: the data script
of type $d$, redeemer script of type $r$, and information of type $P$
about the pending transaction (see Section~\ref{sec:scripts}).  If the
script terminates without producing an error, the input-output pair is
valid. The types $d$ and $r$ vary from script to script, and they are
not relevant for our model. $P$ on the other hand is the same for all
pay-to-script transactions on the blockchain, and it plays a crucial
role in determining the kinds of contracts we can write.

$P$ is the only way for a validator script to see beyond its own
input-output pair and get information about the transaction that it is
part of, and about the blockchain.\footnote{Note that $P$ subsumes
  $\mathtt{blockheight}$ and related primitives in the spec} That has
two implications. First, for any definition of $P$ there should be a
map $\sigma_P : \mathsf{Ledger} \times \mathsf{Transaction} \times
\mathsf{Input} \rightarrow \mathsf{Script}$ that generates a script
with a $P$ value from the ledger, the current transaction, and the
input that is being validated. Second, by varying $P$ -- and hence the
information available to validator scripts -- we can directly change
the kinds of contracts that can be expressed in \textsc{Plutus}. For
example, if we decide that $P = ()$, the unit type, then our validator
scripts would not get any data about the wider transaction at all. In
this setting even our crowdfunding campaign would be impossible to
write because it requires the script to know the total value of all
contributions, and hence the total value of all inputs to the
transaction.  On the other hand, if $P$ is too large (for example, if
it requires searching far back in the history of the chain), then it
could be expensive to calculate.  Parameterising things over $P$
allows us to explore these sorts of issues in a unified way.

In the remainder of this section we are going to use the type
$\mathsf{Script}$ for all scripts, ignoring the Plutus-language types
$d$, $r$ and $P$, and focusing on the blockchain information
generating function $\sigma$ (dropping the subscript).

\subsubsection{Notation}
We generally follow the notation established by
\cite{Zahnentferner18-UTxO}. Types are typeset in
$\mathsf{sans~serif}$. All scripts are of type $\mathsf{Script}$.  The
only operation on $\mathsf{Script}$ from the ledger's perspective is
applying a validator script to three arguments, which is denoted by
$\llbracket \cdots \rrbracket : \mathsf{Script} \rightarrow
\mathsf{Script} \times \mathsf{Script} \times \mathsf{Script}
\rightarrow \mathbb{V}$ where $\mathbb{V}$ is some type containing a
distiguished value \textsf{error} (we could equally well use the type
of booleans, but the version involving \textsf{error} fits Plutus Core
better).

A record data type with fields $\phi_1, \ldots, \phi_n$ of types $T_1,
\ldots, T_n$ is denoted $(\phi_1 : T_1, \ldots, \phi_n : T_n)$. If $t$
is a value of a record data type $T$ and $\phi$ is the name of a field
of $T$ then $t.\phi$ denotes the value of $\phi$ for $t$. A list
$\lambda$ of type $\mathsf{List[}T\mathsf{]}$ is either the empty list
$[]$ or a list $e :: \lambda'$ with $head$ $e$ of type $T$ and $tail$
$\lambda'$ of type $\mathsf{List[T]}$. We denote the $i$-th element of
a list $L$ by $L[i]$.  The concatenation of two lists $\lambda_1$ and
$\lambda_2$ is denoted $\lambda_1 ::: \lambda_2$. $x \mapsto f(x)$
denotes an anonymous function.  A cryptographic collision-resistant
hash of a value $c$ is denoted $c^{\#}$.  If any operation fails (for
example, indexing a list out of range or trying to spend an output
which has already been spent) then we assume that the containing script
returns \textsf{error} immediately.



We have the following basic types, all of which we regard as aliases for
the type of natural numbers:
\begin{itemize}
\item \textsf{Value}: a currency value
\item \textsf{SlotNumber}: a slot number
  \item \textsf{Address}: the address of an object in the blockchain
\item \textsf{TxId}: the identifier of a previous transaction on the chain.
\end{itemize}

\noindent In practice an \textsf{Address} will usually be a hash of
some object (for example, a script), and the blockchain will provide
an efficient way to retrieve the original object given its hash.
Similarly, a \textsf{TxId} will be the hash of a transaction.
Accessing objects indirectly via addresses is helpful because it can
help to reduce memory and disk usage: for example, there may be
scripts for common validation scenarios used in many transactions, and
it is more efficient to store single copies of such scripts rather
than having hundreds of transactions each with their own copy.

\subsubsection{The Definition of Extended UTXO}

The definitions in this section are essentially the definitions of
UTXO-based cryptocurrencies with scripts \cite{Zahnentferner18-UTxO},
with the following additions:
\begin{itemize}
\item Every transaction now has a validity interval (see above).
\item Every output has an asssociated datascript.
\item The notion of validity (Definition
  \ref{def:validity})is now parameterised by $\sigma$, the
  function which provides state information.
\end{itemize}

\newcommand{\mi}[1]{\mathit{#1}}
\newcommand{\inputs}{\mathit{inputs}}
\newcommand{\outputs}{\mathit{outputs}}
\newcommand{\forge}{\mathit{forge}}
\newcommand{\fee}{\mathit{fee}}
\newcommand{\addr}{\mathit{address}}
\newcommand{\val}{\mathit{value}}

\newcommand{\slotnum}{\mathsf{SlotNumber}}
\newcommand{\spent}{\mathsf{spentOutputs}}
\newcommand{\unspent}{\mathsf{unspentOutputs}}
\newcommand{\xutox}{\mathsf{XUtxoTx}}

\noindent\begin{definition}[Transaction]
The datatype for script-address UTXO-based transactions is defined as 
  \begin{align*}
    \xutox = ( &\inputs: \mathsf{Set[Input]},\\
    &\outputs: \mathsf{List[Output]},\\
    &\mathit{validityInterval}: \slotnum \times \slotnum,\\
                        &\forge: \mathsf{Value}, \fee: \mathsf{Value})\\
  \end{align*}

\noindent Note that we have a \textsf{Set} of inputs but a
\textsf{List} of outputs; we use a list because any subsequent
transaction attempting to use an output $O$ as an input will have to
be able to determine exactly where $O$ came from.


\noindent  The datatype for $outputs$ is:
  \[
      \mathsf{Output} = (\addr: \mathsf{Address}, \val: \mathsf{Value}, \mi{datascript}: \mathsf{Script})
    \]
  Output references are
  \[
    \mathsf{OutputRef} = (\mi{id}: \mathsf{Id}, \mi{index}: \mathsf{Int})
  \]
  and inputs
  \begin{align*}
    \mathsf{Input} = (& \mi{outputRef}: \mathsf{OutputRef},\\
                      & \mi{validator}: \mathsf{Script},\\
                      & \mi{redeemer}: \mathsf{Script})\\
  \end{align*}
\end{definition}


\noindent The type $\mathsf{Ledger}$ is then a list of $\xutox$. From
\cite{Zahnentferner18-UTxO} we also get the function

\[
  \unspent : \mathsf{Ledger} \rightarrow \mathsf{Set[OutputRef]}
\]

\noindent given by

\begin{align*}
   \unspent([]) &=\emptyset \\
   \unspent(t:::\lambda) &= (\unspent(\lambda) \setminus \spent(\lambda)) \\
    & \qquad \cup \unspent(t)
\end{align*}

\noindent where $\spent(t) = t.\inputs$ for $t \in \xutox$.

Finally, we define transaction validity for a given $\sigma$. Our definition
\ref{def:validity} combines Definitions 6 and 14 from
\cite{Zahnentferner18-UTxO}, differing from the latter in condition
\ref{all-inputs-validate}. Here we include the blockchain-information
generating function $\sigma : \mathsf{Ledger} \times
\mathsf{Transaction} \times \mathsf{Input} \rightarrow
\mathsf{Script}$.

\begin{definition}[$\sigma$-Validity]\label{def:validity} A transaction $t$ is $\sigma$-\emph{valid} for a ledger $\lambda$ if the following conditions hold:
  \begin{enumerate} 
    \item \label{all-inputs-refer-to-unspent-outputs} \textbf{all
      inputs refer to unspent outputs:}
      \[
        \forall i \in t.\inputs: i \in \unspent(\lambda)
      \]
    \item \label{value-is-preserved} \textbf{value is preserved}:
    \[
      t.\forge + \sum_{i \in t.\inputs} t.\mi{value}(i, \lambda) = t.\fee + \sum_{o \in t.\outputs} o.\mi{value}
    \]
    \item \label{no-double-spending} \textbf{no output is double spent:}
    \[
      |t.\inputs| = |\mi{map}\, (i \mapsto i.outputRef)\, (t.\mi{inputs})|
    \]
    \item\label{all-inputs-validate} \textbf{all inputs validate:}
    \[
    \forall i \in t.\inputs,\enspace \llbracket
    i.\mi{validator}(\mi{out}(i, \lambda).\mi{datascript},i.\mi{redeemer}, \sigma(\lambda, t, i)) \rrbracket \ne \textsf{error}
      \]
    \item\label{validator-scripts-hash} \textbf{validator scripts hash to their output addresses:}
    \[
      \forall i \in t.\inputs,\enspace i.\mi{validator}^{\#} = \mi{out}(i, \lambda).\mi{address}
    \]
  \end{enumerate}
\end{definition}

In practice, $\sigma$-validity imposes a limit on the size of the
$\mathsf{Script}$ values of the $validator$, $redeemer$ and
$datascript$ fields and the result of $\sigma$. The validation of a
single transaction must take place within one slot, so the evaluation
of $\llbracket ~ \rrbracket$ cannot take longer than one slot,
approximately twenty seconds.

By parameterising transaction validity by the blockchain information
generating function $\sigma$, we obtain an equivalence class of
transactions that are indistinguishable by validator scripts with
access to the result of $\sigma$.

%% \section{More stuff}

%% When $T$ wants to unlock $O$,
%% the redeemer for the appropriate input of $T$ is applied to three arguments:
%% \begin{itemize}
%% \item A \textit{datascript}, provided by the producer $S$.
%% \item A \textit{validator} script, provided by the producer $S$. 
%% \item A \textit{redeemer} script, provided by the consumer $T$.  
%% \end{itemize}

%% \blue{The GitHub page on Extended UTXO at
%%   https://github.com/input-output-hk/plutus/tree/master/docs/extended-utxo
%%   says ``The validator script must be submitted as part of the
%%   consuming transaction's input, but its content is determined by the
%%   producing transaction.''. I'm finding this difficult to explain.}


%%  When $T$ wants to unlock $O$, the validator for $O$ of $T$ is applied
%%  to three arguments: the redeemer script, the data script, and
%%  information (supplied by the slot leader) about the current
%%  transaction and the state of the blockchain, described by some type
%%  $P$.

%% If the result evaluates to anything other than the Plutus Core
%% \texttt{error} term then validation succeeds and the output is unlocked.

%% Every input of a transaction has an associated redeemer and for the
%% transaction to proceed the redemption of every input must be validated
%% successfully.

%% In the current Plutus implementation, the type $P$ contains the
%% following information:

%% \begin{itemize}
%% \item The validity interval of $T$.
%% \item The hash of $T$.
%% \item For every input of $T$, its value and the hashes of its
%%   validator, data, and redeemer scripts (if the input refers to an output
%%   locked by a script) or the public key and signature (if it refers to
%%   an output locked by a public key)
%% \item For every output of $T$, its value and the hash of its validator
%%   and data script, or the public key that owns it
%% %\item The sum of the values of all unspent outputs (of the current
%%   %blockchain without $T$) which are locked by $O$'s validator script.
%% \end{itemize}

%% The expressivity of contracts depends on the choice of $P$, and $P$
%% may be extended in future to include more information if required.

%% \subsection{Extended UTXO and Standard UTXO}

%% \blue{This is really feeble.}

%% Our extended version of the UTXO model genuinely does offer more than
%% the standard version, as implemented by Bitcoin.

%% %% Duncan: Perhaps one approach to explaining this is to start by
%% %% noting that classically we don't think of data values flowing along
%% %% the edges of the utxo graph, but that this is very useful for
%% %% scripts.

%% \subsubsection{Datascripts}
%% \label{sec:datascripts}
%% Datascripts allow us to separate information about the state
%% of the transaction from the validator script, meaning that we can
%% have situations such as a single validator which refunds outputs
%% to addresses contained in the data script (as in the crowdfunding
%% example).  In the Bitcoin model, each output would require a
%% specialised validator, increasing memory usage.
    
%% \subsubsection{Constraining outputs}
%% \label{sec:constraining-outputs}
%% The fact that the validator receives information about the
%% state of the chain allows us to do things like restricting where
%% the output of a transaction goes, so it's possible to construct
%% graphs of transactions where values are constrained to flow down
%% allowed paths. In the classic model if you can validate at all you
%% can always take the money and keep it.

%% \subsubsection{Deterministic Scripts}
%% \label{sec:deterministic-scripts}
%% When core nodes validate an input spending from a pay-to-script
%% address, they supply two pieces of information to the validator
%% script: The structure of the transaction that is being validated (in
%% form of a list of hashes and signatures), and an indication of the
%% current time (expressed in terms of the slot number: see
%% \ref{subsection:blockchains}).  After a transaction has been submitted
%% to the blockchain, it may linger in the transaction pool for some time
%% before being processed by a core node. Allowing validator scripts to
%% know the exact slot number of their validation would therefore make
%% them non-deterministic, in the sense that their result depends on
%% information that is not known at the time the transaction is
%% submitted.  Non-determinism is a problem because it makes it
%% impossible to accurately compute the gas cost of consuming a
%% pay-to-script output, running counter to our stated goals of
%% measurable and predictable resource consumption.

%% In the Extended UTXO model we add a \emph{validity interval}, an
%% interval of slots, to each transaction's metadata. Core nodes only
%% attempt to validate transactions whose validity interval contains the
%% current slot. When a scripted transaction input is validated, the
%% transaction's validity interval is passed to the validator script,
%% providing information about the current time. Since the validity
%% interval is known at the time the transaction is submitted to the
%% chain, transaction validation in Plutus is completely deterministic.

%% As a result, the exact amount of gas that is required to run the script can be 
%% calculated in advance (by running it), and users do not risk being surprised by 
%% failed validations that still incur fees.    


\bibliographystyle{plain} %% ... or whatever
\bibliography{extended-utxo-specification}


\end{document}
